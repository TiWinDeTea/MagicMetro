\documentclass[report, backcover, french, nodocumentinfo]{upmethodology-document}
\usepackage[utf8]{inputenc}
\usepackage[T1]{fontenc}
%\usepackage{natbib}
\usepackage{csquotes}
%\usepackage[backend=biber,style=alphabetic,citestyle=authoryear]{biblatex}
\usepackage{amsmath}
\usepackage{amssymb}
\usepackage{hyperref}
\usepackage{listings}
\usepackage{textcomp}
\usepackage{color}
\usepackage[toc,page]{appendix}
\usepackage{wrapfig}
\usepackage{stackengine}
\usepackage{scalerel}

%link option, especialy for the table of contents
\hypersetup{
    colorlinks=true,
    linkcolor=black,
    urlcolor=blue,
    linktoc=all
}

\setfrontcover{classic}

\declaredocument{Conception de projet: clone de MiniMetro}{Rapport LO43}{LO43-A2016}
\setpublisher{Université de Technologie de Belfort-Montbéliard}

%\incversion{\makedate{jj}{mm}{aaaa}}{Initial version.}{\upmpublic}
\incversion{\today}{Initial version.}{\upmpublic}

\addauthorvalidator*[julien.barbier@utbm.fr]{Julien}{BARBIER}{Auteur original}
\addauthorvalidator*[maxime.pinard@utbm.fr]{Maxime}{PINARD}{Auteur original}

\addinformed*[franck.gechter@utbm.fr]{Franck}{GECHTER}{Professeur de l'UV LO43}

\setdockeywords{UTBM, LO43, MiniMetro, Java, JavaFX, UML}

\setcopyrighter{Julien BARBIER et Maxime PINARD}
\setpublisher{Julien BARBIER et Maxime PINARD}
\setprintingaddress{France}

%\setfrontcover{modern}
%\setfrontillustration[0.6]{figures/logo}

\graphicspath{./figures/}

%\bibsize{\normalfont}

\newcommand*\cleartoleftpage{%
  \clearpage
  \ifodd\value{page}\hbox{}\newpage\fi
}

\newcommand{\p}[1]{\paragraph{#1\\}}

%Function to print a warning sign
\newcommand{\dangersign}[1][2.5ex]
	{\renewcommand{\stacktype}{L}
		{\scaleto{\stackon[1pt]{\color{red}$\triangle$}{\fontsize{4pt}{4pt}\selectfont !}}{#1}}}

\frenchbsetup{StandardLayout=true,ReduceListSpacing=false,CompactItemize=false}

% For more information about UPmethodology: https://www.ctan.org/pkg/upmethodology

\begin{document}

	\upmdocumentsummary{}
	\upmdocumentauthors{}
	\upmdocumentinformedpeople{}
	\upmpublicationpage{}

	\tableofcontents{}
	\listoffigures{}

	\newpage{}
	\chapter{Présentation de Mini Metro}
		\section{Un peu d'histoire...}
			\p{}
				Mini Metro est un jeu développé par le studio indépendant Dinosaur Polo inc. Il a été présenté sous le nom de Mind The Gap à la 26ème édition du Ludum Dare qui a eu lieu le 26 avril 2013. Il est ensuite développé pour devenir un jeu complet et est proposé aux Steam Greenlight qui permet au public de choisir quel jeu indépendant va entrer dans le catalogue Steam de manière permanente ou temporaire. Il sort sous sa forme définitive sur Steam et GOG.com le 6 novembre 2015. Il est ensuite porté sur les plateformes mobiles Android et IOS le 18 octobre 2016.
		\section{But du jeu}
			\p{}
				Mini Metro est un jeu de gestion de métro. Des trains transportent des passagers de station en station en suivant des lignes. On doit gérer les lignes de métro pour pouvoir desservir toutes les stations. Ces stations se remplissent, au fur et à mesure du temps, de passagers. Les passagers décident de leur destination en fonction de la forme géométrique de la station. Par exemple, un passager arrive à une station triangle et souhaite aller dans une station carrée. Il faut donc relier les stations par des lignes et ainsi éviter qu’une station surcharge. Si une station surcharge, la partie se termine.
			\p{}
				Le jeu propose plusieurs villes correspondantes à un niveau de difficulté et de nouveaux trains font leur apparition.
		\section{Analyse du jeu}
			\p{}
				Au démarrage du jeu, un menu succint offre trois possibilités : jouer, modifier ou quitter. Si le joueur veut :
				\begin{itemize}
					\item jouer. Il a alors accès à un autre menu qui lui propose toutes les villes disponibles et le joueur fait son choix.
					\item entrer dans les paramètres. Il pourra modifier le volume et la couleur du fond (blanc ou gris foncé).
					\item quitter le jeu.
				\end{itemize}
			\p{}
				A l'intérieur d'une ville, le joueur peut créer des lignes entre les stations. Il possède un inventaire dans lequel se trouve au départ 3 lignes, 3 trains et 2 tunnels. Le jeu est en temps réels ainsi tous les éléments du jeu sont fonctions du temps (l'ajout de station, de passagers...). Tous les dimanches, le joueur reçoit un bonus fixé par le jeu de façon aléatoire et parfois en plus un choix possible entre deux bonus.
			\p{}
				Le joueur peut modifier en temps réel les lignes voir en retirer. Si elle est supprimée, la ligne retourne dans l'inventaire du joueur. La modification ou suppression d'une ligne est sans incidence sur le trajet d'un train se trouvant sur celle-ci jusqu'à son arrêt à la station finale. \\Si une section de la ligne passe dans une partie de l'eau, alors cette section devient un tunnel et on a un nombre limité de tunnel.
			\p{}
				Le joueur peut aussi ajouter ou retirer un train à une ligne. Si le train est retiré, celui-ci retourne dans l'inventaire. Le joueur peut déplacer un train, mais celui-ci reste immobile pendant 2-3 secondes et les passagers présents avant le déplacement retournent à la station d'origine.

	\chapter{Conception globale du projet}
		\section{Cas d'utilisation}\label{useCases}
			\begin{figure}[h!]
				\centering
				\includegraphics[width=\textwidth]{figures/UseCaseDiagram}
				\caption{Diagramme des cas d'utilisation}
				\label{fig:UseCaseDiagram}
			\end{figure}
			\p{Simulateur}
				MiniMetro a pour particularité d'avoir un fonctionnement proche d'un simulateur, en effet à l'inverse d'un jeu d'échec par exemple qui attend sur l'utilisateur pour effectuer des actions, MiniMetro lui a un fonctionnement en temps réel indépendant de l'utilisateur mais dépendant de l’environnement, que l'utilisateur peut modifier. En effet si l'utilisateur ne fait rien, le jeu continue mais cela mènera vite a la défaite, par conséquent l'utilisateur modifie l’environnement (trace des lignes, ajoute de wagons aux trains\ldots) pour empêcher la ``simulation'' d'aller à sa perte.
			\p{Actions de l'utilisateur}
				Les actions de l'utilisateur se répartissent en trois catégories:
				\begin{itemize}
					\item Choix (avant le début de la partie):
						\begin{itemize}
							\item De la carte
							\item Des options de jeu
						\end{itemize}
					\item Ajustement de la vitesse de jeu
					\item Déplacements d'éléments
						\begin{itemize}
							\item De la carte à l'inventaire
							\item De l'inventaire à la carte
							\item D'une position à l'autre (trains)
							\item Même modifier une ligne revient à déplacer les connexion et sous-sections de cette ligne\footnote{La représentation des lignes est expliquée ultérieurement}
						\end{itemize}
				\end{itemize}
				Les actions décisives possibles une fois en jeux pour l'utilisateur sont donc toutes des déplacements.
		\section{Le modèle MVC}
			\subsection{Adaptation à notre utilisation}
				\p{Choix}
					Nous avons choisi de séparer notre projet en 3 parties: modèle, vue et contrôleur (MCV) pour organiser le code et rendre chaque partie plus indépendante des autres. Ceci dans le but d'obtenir un code maintenable et facilement adaptable a une autre bibliothèque graphique par exemple puisqu'il suffirait de ré-implémenter une partie de la vue.
				\p{Modèle}
					Il possède toute la logique du jeu il décide à chaque instant de quoi faire et s'adapte en fonction de l’environnement. Environnement qu'il maintient à jour en fonction des modifications dont il est notifié.
				\p{Vue}
					Elle est composée de tous les éléments graphiques du projet et est donc directement dépendante de la bibliothèque graphique utilisée à l'implémentation. Nous utiliserons \textit{JavaFx}.
				\p{Contrôleur}
					Il s'occupe de donner un sens à la vue, transforme les actions abstraites du joueur (glisser sa souris de la position $(14,35)$ a la position $(452,128)$) en action concrète pour le modèle (tracer une ligne de la station $A$ a la station $B$). Il s’occupe aussi le l'action inverse de transformer les actions du modèle (déplacer le train) en actions visuelles (déplacer la vue associée au train).
				\p{Prise en compte de JavaFx}
					Puisque notre vue est réalisée par une librairie graphique externe au projet (ici \textit{JavaFx}) nous avons décidé de fusionner le contrôleur et la vue au niveau de la répartition des packages, le package \jpackage{view} regroupant donc les classes de contrôle qui en interne utilisent les éléments de \textit{JavaFx}. On obtient donc deux packages principaux: \jpackage{modele} et \jpackage{view}.
				\p{}
					On arrive donc sur un modèle hybride entre modèle vue (MV) et modèle, vue et contrôleur (MCV) du fait que la séparation contrôleur -- vue qui n'apparait pas au niveau des package, cela étant du a l'utilisation d'une librairie graphique externe au projet. Néanmoins notre conception n'est pas dépendante de \textit{JavaFx} et une autre librairie graphique pourrait très bien être utilisée.

	\chapter{Classes et liaisons}
		\section{Modèle}
			\begin{figure}[h!]
				\centering
				\includegraphics[width=1\textwidth]{figures/ModelClassDiagram}
				\caption{Diagramme de classes: Modèle}
				\label{fig:ModelClassDiagram}
			\end{figure}
			\subsection{Problématiques}
				\p{}
					Pour faire le modèle, nous nous sommes basés sur les éléments visibles à l'écran (trains, lignes, stations\ldots voir diagramme \ref{fig:ModelClassDiagram}). Un des problèmes que nous avons rencontré concerne les lignes. Pour pouvoir faire des modifications sur la ligne, nous avons décidé de diviser les lignes en sections qui correspondent aux rails qui relient deux stations entre elles. Une section peut être composée d'un angle dont on doit connaitre les coordonnées. Nous avons ainsi des sous-sections au nombre maximum de deux par section. Ces sous-sections peuvent être des tunnels ou non.
			\subsection{Lignes}
				\begin{figure}[h!]
					\centering
					\includegraphics[width=1\textwidth]{figures/LinesClassDiagram}
					\caption{Diagramme de classes: Lignes}
					\label{fig:LinesClassDiagram}
				\end{figure}
				\p{Connection}
					Comme on peut le voir sur le diagramme \ref{fig:LinesClassDiagram}, nous avons un package \jpackage{lines} qui contiendra toutes les classes qui décriront les lignes qu'on peut voir dans le jeu. Les connexions correspondent aux point qui séparent les sous-sections au sein de sections. Ces connexions peuvent être au sein d'une station.
				\p{Sous-section}
					Une sous-section est composée de 2 connections qui la délimitent pour connaire sa longueur. On retiendra avec un booléen si la sous-section est un tunnel ou pas pour pouvoir décrémenter ou non le compteur de tunnel présent dans l'inventaire et compter ceux-ci. On retiendra aussi la longueur de la sous-section.\\
					Dans la classe \jclass{Section}, on aura le booléen hasTunnel qui permettra de savoir si la section possède un tunnel ou pas. On aura aussi sa longueur et les connexions d'arrivée et de départ pour délimiter celle-ci. On retiendra aussi la ligne à laquelle appartient la section.
				\p{Ligne}
					Dans la classe \jclass{Line}, on retiendra les connections des sections qui la composent. On stocke aussi les trains qui sont présents au sein de la ligne. On a décidé de faire une composition avec les sections. En effet, si on décide de retirer une ligne, on aura plus de section, celle-ci étant détruite.
			\subsection{Trains}
				\begin{figure}[h!]
					\centering
					\includegraphics[width=1\textwidth]{figures/TrainClassDiagram}
					\caption{Diagramme de classes: Trains}
					\label{fig:TrainClassDiagram}
				\end{figure}
				Pour les trains, on a la classe \jclass{Train} et \jclass{PassengerCar} où \jclass{Train} est l'objet qui se déplace et le \jclass{PassengerCar} est le wagon. La classe Train est composée de sa vitesse maximale qui va être associée en fonction de l'enum  \jclass{TrainType}. Cette enum permettra d'avoir différents trains en fonction du niveau. Le train possédera une ligne s'il est sur la carte. Un  \jclass{Train} possède forcément au moins un  \jclass{PassengerCar} qui est l'unité qui transporte les personnes (les wagons). Nous sauvegarderons aussi sa position à un temps donné.\\
				Le  \jclass{Train} possèdera aussi son état. L'état du train est une interface nommée \jinterface{TrainState} qui sera implémentée par \jinterface{MovingState} et par \jinterface{AtStationState}. Ainsi, quand on sera dans une station, l'état du train sera AtStationState et quand ils se déplacera sur la ligne, son état sera une  \jclass{MovingState}. Dans \jinterface{AtStationState}, le \jclass{Train} déposera les passagers qu'il transporte et qui veulent aller à la station présente, et prendra les passagers de la station en fonction de sa capacité. Dans \jinterface{MovingState}, le \jclass{Train} se déplacera en fonction de la ligne sur laquelle il est affecté. Ce choix a été fait pour qu'on appelle la fonction live dans la boucle de jeu sans se soucier de l'endroit où est le train.
			\subsection{Stations}
				La classe \jclass{Station} (ou gare) va contenir les passagers qui veulent aller à une autre station, sa capacité maximum, les connections qu'elles possèdent pour le croisement de lignes et son type qui est ici représenté sous la forme d'un enum appelé \jclass{StationType}.
			\subsection{Réunificaiton avec le GameManager}
				Ces différentes classes vont être manipulées par un objet appelé \jclass{GameManager}. Celui-ci va gérer le jeu et définir la boucle de jeu.
				\subsubsection{la boucle de jeu}\label{gameLoop}
					La boucle de jeu est ce qui va simuler le temps réel. Pour ce faire on va appeler celle-ci un certain nombre de fois fixe par secondes et elle regardera le \jclass{TimeManager} pour savoir si il faut rafraichir ou pas. Dans cette boucle, on vérifie si on a une action à faire en consultant le \jclass{MapScript}. S'il y a une action à effectuer, il applique ce qui est précisé dans le \jclass{MapScript} (apparition d'un bonus, d'une station\ldots). Sinon, on ne fait rien et on continue. Ensuite, on applique à tous les trains présents dans la map la fonction \jfunc{live} qui permettra de mettre à jour les trains.
				\subsubsection{Path-finding}
					La \jclass{GameMap} va contenir tous les objets qui seront présents sur la map (stations, lignes, trains\ldots). C'est cet objet qui va calculer le path-finding pour chaque passager qui va apparaitre à une station donnée. Celui-ci possédant une référence sur tous les objets présents sur la map, il pourra trouver le chemin optimal pour le passager en question.
		\section{Vue}
			La vue n'a pas encore de conception définitive mais son fonctionnement global a déjà été choisi.
			\subsection{Gestion centralisée}
				Notre vue sera gérée par le \jclass{ViewManager}. Ce dernier est celui qui communique avec le modèle (communication détaillé dans les sections suivantes). C'est lui qui instancie les éléments de la vue en fonction de la skin qui lui aura été passé en paramètre de constructeur. Cette skin (dont la conception reste encore à définir) est le groupe d'objets graphiques qui sera utilisée pour représenter les éléments à l'écran, la skin de base étant celle de \textit{MiniMetro} avec des rectangles colorés pour les trains et des formes (triangle, rond, \ldots) pour les stations. Lorsque le \jclass{GameManager} demande une vue pour un train il passe en paramètre le type du train, ainsi le \jclass{ViewManager} crée une vue du type associé avec la skin qui correspond au type du train. Lorsque le train est retiré de la carte, on passe par le \jclass{ViewManager} pour retirer la vue du train de la carte et mettre à jour le compteur de train de l'inventaire. C'est donc le \jclass{ViewManager} qui instancie et gère tous les éléments de la vue.
			\subsection{Gestion déléguée}
				Une fois la référence sur la vue fournie au train ou à la station par le biais de l'interface, sa gestion est déléguée au train ou station qu'elle représente. Les fonctions de l'interface permettant par exemple pour une station d'ajouter des passagers en précisant juste le type de station vers laquelle ce dernier veut aller et l'implémentation de la vue se charge de prendre le skin associé pour ajouter graphiquement un passager.
			\subsection{Contrôle utilisateur}
				Le jeu se joue entièrement à la souris et les interactions de l'utilisateur. Comme expliqué dans les cas d'utilisation (voir \ref{useCases}) une fois en jeux il y plusieurs types d'interaction: le contrôle du temps à l'aide de boutons, le choix des bonus lorsqu’ils apparaissent et le déplacement d'éléments du jeu.
				\p{Contrôle du temps}
					L'interface graphique contiendra un slider pour régler la vitesse du jeu, un bouton lecture et un bouton pause. A la modification de la vitesse de jeu ou la mise en pause / reprise le \jclass{ViewManager} utilise les fonction du \jclass{TimeManager} pour modifier le temps. (voir section \ref{timeControl})
				\p{Choix des bonus}
					Pour le choix d'un bonus le \jclass{GameManager} appelle une fonction du \jclass{ViewManager}. Le jeu est en pause le temps que le joueur choisisse un bonus, la fonction peut donc être bloquante. Au cas où nous aurions besoin de continuer à exécuter des actions dans le thread du \jclass{GameManager}, nous réaliserons une fonction non bloquante qui retourne un \jclass{Futur} ou notifierons le \jclass{GameManager du choix} par un event.
				\p{Déplacement d'éléments}
					Pour le déplacement des éléments nous avons décidé d'utiliser des états à la façon d'une state machine visible sur la figure \ref{fig:ModificationStatesClassDiagram}. Le \jclass{ViewManager} aura une référence sur un \jclass{ModificationState}, au clic de la souris en fonction de l'objet pointé, on l'assignera à l'état qui correspond (\jclass{MovingTrainState} si c'était un train\ldots) que l'objet soit sur la carte ou dans l'inventaire. On appelle aussi la \jfunc{init} pour initialiser l'état (faire apparaitre le train sous la souris\ldots) et instancier les objets graphiques temporaires qui vont permettre une visualisation de la modification par l'utilisateur. Puis tout le long du drag n drop la fonction \jfunc{update} sera appelée pour actualiser ces objets temporaires en fonction (déplacer le train, tracer la ligne jusqu’à la souris\ldots). Au relâchement de la souris la fonction \jfunc{apply} est appelée pour que les objets temporaires soient supprimés, qu'un event soit envoyé et l'action validée par la logique qui modifiera la vue en conséquence (ajouter un train a une ligne\ldots).
					\begin{figure}[h!]
						\centering
						\includegraphics[width=\textwidth]{figures/ModificationStatesClassDiagram}
						\caption{Diagramme de classes: Modification states}
						\label{fig:ModificationStatesClassDiagram}
					\end{figure}
		\section{Liaison modèle -- vue}
			\begin{figure}[h!]
				\centering
				\includegraphics[width=\textwidth]{figures/ModelViewLinkClassDiagram}
				\caption{Diagramme de classes: Liaison modèle -- vue}
				\label{fig:ModelViewLinkClassDiagram}
			\end{figure}
			\p{Choix}
				Nous avons décidé pour pouvoir faire communiquer le modèle vers la vue de faire une communication par interface. Cette solution permet de faire une séparation entre le modèle et la vue. Elle permet aussi une certaine indépendance avec l'implémentation de la vue.
			\p{Fonctionnement}
				Comme on peut le voir avec le diagramme UML \ref{fig:ModelViewLinkClassDiagram}, les \jclass{Trains} et les \jclass{stations} vont avoir une référence respectivement sur une \jinterface{TrainView} et une \jinterface{StationView}. Ces deux interfaces sont implémentées dans le package view permettant aux modèles d'intéragir avec la vue. De même pour les autres classes comme \jclass{SubSection}, \jclass{Inventory} et \jclass{GameMap}.
		\section{Liaison vue -- modèle}
			\subsection{Communication par event}
				\p{Choix}
					Nous avons décidé que la communication entre la vue et le modèle se ferait par event de façon à ce que la vue n'ait pas de référence sur les éléments du modèle. De cette façon on évite les problèmes de référencement circulaire qui en Java peuvent empêcher le \texttt{grabage collector} de libérer la mémoire. De plus cela accentue la séparation entre la vue et le modèle.
				\p{Fonctionnement}
					Le principe des events est que certains objets sont des \jclass{EventListener} (implémentent l'interface en Java), c'est à dire qu'ils possèdent une fonction à appeler pour les notifier d'un event. Cette fonction est souvent nommée \jfunc{onEvent} et elle contient les actions à effectuer lors de la réception d'un event. Tout ceci utilise des template sur les types des \jclass{Event} qui sont des extensions d'une base class \jclass{Event}. En java on utilisera les \texttt{Generics}.
			\subsection{EventDispatcher}
				\p{}
					Nous avons décidé d'utiliser un \texttt{EventDispatcher} pour l'envoi de nos events afin d'éviter le stockage de liste d'\jclass{EventListener} dans toutes nos classes qui envoient des events. Ce dernier a déjà été codé avant le projet, voir figure \ref{fig:EventDispatcherClassDiagram}. Il utilise des \jclass{WeakReference} et n'est donc pas a prendre en compte dans la gestion de la mémoire. De plus il est thread safe.
					\begin{figure}[h!]
						\centering
						\includegraphics[width=\textwidth]{figures/EventDispatcherClassDiagram}
						\caption{Diagramme de classes: EventDispatcher}
						\label{fig:EventDispatcherClassDiagram}
					\end{figure}
				\p{Fonctionnement}
					L'\jclass{EventDispatcher} est un singleton, il peut être utilisé de n'importe où avec \jfunc{getInstance}. Les méthodes \jfunc{addListener} et \jfunc{removeListener} permettent de s'enregistrer ou dé-enregistrer en tant que \jclass{EventListener} pour un certain type d'\jclass{Event} passé en paramètre. Pour cela il faut implémenter l'interface \jinterface{EventListener}. Depuis \textit{Java 8} il est aussi possible de passer en paramètre une fonction lambda.
			\subsection{Utilisation concrète}
				\p{}
					La figure \ref{fig:ViewModelLinkClassDiagram} représente la liaison entre la vue et le modèle avec les events qui notifient des modifications sur certains éléments du jeu à titre d'exemple. Différent types d'events seront créés pour chaque action de l'utilisateur mais la conception de la vue n'étant pas encore définitive et dépend de \textit{JavaFx} nous n'avons pas encore décidé d'une représentation des données à envoyer dans les events (ex: description de la nouvelle position d'une ligne qui a été modifié).
					\begin{figure}[h!]
						\centering
						\includegraphics[width=\textwidth]{figures/ViewModelLinkClassDiagram}
						\caption{Diagramme de classes: Liaison vue -- modèle}
						\label{fig:ViewModelLinkClassDiagram}
					\end{figure}
				\p{}
					Concrètement et comme visible sur la figure \ref{fig:ViewModelLinkClassDiagram}, le \jclass{GameManager} possède des \jclass{EventListener} qui correspondent à chaque type d'event envoyé par le \jclass{ViewManager}, ces \jclass{EventListener} feront appel aux méthodes \jfunc{onEventXXX} du \jclass{GameManager}. La réception des events dans le modèle se fait donc par le \jclass{GameManager} qui modifie le contexte en conséquence. Du côté de la vue, sa conception n'étant pas encore définitive et dépendante de \textit{JavaFx}, il se pourrait que les events ne soient pas tous envoyés par le \jclass{ViewManager} mais qu'il soient envoyés par les éléments directement concernés (par exemple le \jclass{TrainView} pour notifier de son déplacement).

	\chapter{Solutions particulières}
		\section{Gestion du temps}
			\subsection{TimeManager}
				Pour résoudre le problème de la gestion du temps (lecture, pause, changement de vitesse) dans le jeux nous avons décidé de créer un \jclass{TimeManager}, cette classe sera un \textit{singleton} qui possèdera son propre thread. Le \jclass{TimeManager} aura une variable qui représente le temps écoulé depuis son démarrage. Il sera possible d'obtenir ce temps en secondes (pour l'affichage dans la vue) ou en millisecondes (pour le modèle). Là où l'utilisation du \jclass{TimeManager} devient intéressant plutôt que d'utiliser le temps système c'est que sa vitesse d'écoulement du temps sera réglable. Sa variable de temps augmentera donc plus ou moins vite en fonction de la vitesse d'écoulement du temps précisé au \jclass{TimeManager}.
			\subsection{Temps réel}
				Le moteur du jeu, la boucle principale (déja abordée à la section \ref{gameLoop}) sera dans un thread géré par le \jclass{GameManager} et tournera à une vitesse fixe pour maintenir une fluidité. Pour les stations et bonus, à chaque tour de boucle on regarde le timecode de la prochaine apparition et on le compare au timecode actuel du \jclass{TimeManager}, si celui-ci est supérieur à celui de l'apparition on applique cette apparition. Pour les trains, il y aura un nombre fixe d'appel par seconde du jeu à leur fonction \jfunc{live}, secondes du \jclass{TimeManager} donc. Ainsi le nombre d'appel par secondes réelles (la vitesse du jeu) dépend de l'écoulement du temps du \jclass{TimeManager}, si celui-ci est en pause la fonction \jfunc{live} des trains n'est pas appelée et ceux-ci ne se déplacent donc plus. De même le timecode d'apparition des stations et bonus n'est pas atteint (le temps n'avance pas), le jeu est donc bien en pause.
			\subsection{Contrôle du temps}\label{timeControl}
				La liaison entre le \jclass{ViewManager}, le \jclass{GameManager} et le \jclass{TimeManager} est visible sur la figure \ref{fig:TimeManagerLinkClassDiagram}. Le \jclass{ViewManager} lors de l'utilisation des boutons de contrôle du temps utilise les fonctions du \jclass{TimeManager} pour régler le temps en fonction des modifications de l'utilisateur. Le \jclass{GameManager} lui lit le temps écoulé dans sa boucle de jeux pour effectuer ou non des actions en fonction du temps écoulé. De plus lors de la modification de la vitesse de temps, sa mise en pause ou sa reprise le \jclass{TimeManager} envoie des events pour notifier de ces changements. Ces events sont écoutés par le \jclass{GameManager} qui pourra arrêter sa boucle de jeu pour un \jclass{PauseEvent} et la relancer pour un \jclass{StartEvent} pour ne pas tourner inutilement par exemple (à voir à l'implémentation).
				\begin{figure}[h!]
					\centering
					\includegraphics[width=\textwidth]{figures/TimeManagerLinkClassDiagram}
					\caption{Diagramme de classes: Liaison au TimeManager}
					\label{fig:TimeManagerLinkClassDiagram}
				\end{figure}
		\section{Description des cartes}
			\p{problème}
				Les cartes de MiniMetro sont plus qu'une simple description de zone puisqu'elles sont entièrement scriptées, les bonus qui apparaitront, le moment où ils apparaîtront, les apparitions de stations, leur forme, position\ldots Sont identiques à chaque partie sur la carte.
			\p{Scripts}
				\begin{figure}[h!]
					\centering
					\includegraphics[width=\textwidth]{figures/ScriptsClassDiagram}
					\caption{Diagramme de classes: Scripts}
					\label{fig:ScriptsClassDiagram}
				\end{figure}
				Notre solution, visible sur la figure \ref{fig:ScriptsClassDiagram} a été de créer des classes de description que nous avons nommées scripts. A début d'une partie est créé le \jclass{GameManager}, on passe à ce dernier en paramètre de constructeur un \jclass{MapScript} qui contient toute les informations nécessaires au déroulement de la partie.
			\p{Utilisation}
				La boucle de gameplay principale qui gère principalement les trains s'occupera à chaque tour de boucle de regarder si le timecode du prochain évènement a été dépassé, si c'est le cas il agira en fonction de l'évènement.
			\p{Apparition d'une station}
				Dans le cas où le timecode d'apparition de la prochaine station a été atteint, le \jclass{GameManager} instanciera une nouvelle \jclass{Station} avec les informations du \jclass{StationScript} tel que la position. Il utilisera la fonction \jfunc{createStationView} du \jclass{ViewManager} en passant en paramètre le type de la station pour obtenir une vue avec le skin associé au type de la station, le \jclass{StationView} retourné sera passé à la \jclass{Station}.
			\p{Choix d'un bonus}
				Les bonus dans Minimétro étant toujours des éléments à ajouter à l'inventaire (train, ligne\ldots), choisir un bonus revient juste à choisir un élément à ajouter à son inventaire. Dans le cas où le timecode de la sélection du / des prochain bonus est dépassé, le \jclass{GameManager} utilisera les fonctions (que nous n'avons pas encore définies) du \jclass{ViewManager} pour demander à l'utilisateur de choisir entre les différents bonus décrits dans le \jclass{ElementChoiceScript}. En fonction du choix de l'utilisateur le \jclass{GameManager} sélectionnera les scripts correspondants et suivra la même procédure que pour les \jclass{Station}. Il instanciera les éléments en fonction du script, leur associera une vue et les ajoutera à l'inventaire. L'énumération \jclass{ElementType} permet de sécuriser les cast.
		\section{Menu du jeu}
			Pour faire le menu du jeu nous avons pensé à un système de différents écrans (surement des \jclass{Scene} de \textit{JavaFx}) qui représenteront graphiquement des menus qui contiendront des boutons vers d'autres menus. Le menu fonctionnera sur un système de pile de menus, à chaque fois que l'utilisateur cliquera sur un bouton qui l'amènera sur un nouveau menu (menu principal $\rightarrow$ options $\rightarrow$ affichage\ldots), on pushera le menu actuel dans la pile et on affichera l'écran suivant. Cela permetra une gestion facile des sous-menus imbriqués, lorsque l'utilisateur désirera retourner au menu précédent (touche échap ou bouton retour) il suffira de pull le dernier menu de la pile qui sera le menu précédent. Si l'utilisateur lance une partie, on push le menu actuel dans la pile, ont créé un \jclass{GameManager} qui créé \jclass{ViewManager}, on prend la \jclass{Scene} de ce dernier et on la place à l'écran. La partie commence alors.
	%\theupmdockeywords
	%\makebackcover
\end{document}
