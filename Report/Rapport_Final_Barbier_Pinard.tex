\documentclass[report, backcover, french, nodocumentinfo]{upmethodology-document}
\usepackage[utf8]{inputenc}
\usepackage[T1]{fontenc}
%\usepackage{natbib}
\usepackage{csquotes}
%\usepackage[backend=biber,style=alphabetic,citestyle=authoryear]{biblatex}
\usepackage{amsmath}
\usepackage{amssymb}
\usepackage{hyperref}
\usepackage{listings}
\usepackage{textcomp}
\usepackage{color}
\usepackage[toc,page]{appendix}
\usepackage{wrapfig}
\usepackage{stackengine}
\usepackage{scalerel}
\usepackage{fancyvrb}

%link option, especialy for the table of contents
\hypersetup{
    colorlinks=true,
    linkcolor=black,
    urlcolor=blue,
    linktoc=all
}

\setfrontcover{modern}
\setfrontillustration[1.3]{figures/MiniMetro}

\upmcodelang{java}

\declaredocument{Etat de projet: clone de MiniMetro}{Rapport LO43}{LO43-A2016}

\incversion{\makedate{\the\day}{\the\month}{\the\year}}{Initial version.}{\upmpublic}
%\incversion{\today{}}{Initial version.}{\upmpublic}

\addauthorvalidator*[julien.barbier@utbm.fr]{Julien}{BARBIER}{Auteur original}
\addauthorvalidator*[maxime.pinard@utbm.fr]{Maxime}{PINARD}{Auteur original}

\addinformed*[franck.gechter@utbm.fr]{Franck}{GECHTER}{Professeur de l'UV LO43}

\setdockeywords{UTBM, LO43, MiniMetro, Java, JavaFX, UML}

\setcopyrighter{Julien BARBIER et Maxime PINARD}
\setpublisher{Julien BARBIER et Maxime PINARD}
%\setpublisher{Université de Technologie de Belfort-Montbéliard}
\setprintingaddress{France}

\graphicspath{./figures/}

\newcommand*\cleartoleftpage{%
  \clearpage
  \ifodd\value{page}\hbox{}\newpage\fi
}

\newcommand{\p}[1]{\paragraph{#1\\}}

%Function to print a warning sign
\newcommand{\dangersign}[1][2.5ex]
	{\renewcommand{\stacktype}{L}
		{\scaleto{\stackon[1pt]{\color{red}$\triangle$}{\fontsize{4pt}{4pt}\selectfont !}}{#1}}}

\frenchbsetup{StandardLayout=true,ReduceListSpacing=false,CompactItemize=false}

% For more information about UPmethodology: https://www.ctan.org/pkg/upmethodology

\begin{document}

	\upmdocumentsummary{}
	\upmdocumentauthors{}
	\upmdocumentinformedpeople{}
	\upmpublicationpage{}

	\newpage{}
	\section{Etat général}
		\p{}
			Le projet utilise \textit{Apache Maven} pour compiler, un \textit{jar} classique et un contenant les dépendances peuvent être créés avec la commande:
			\begin{Verbatim}[frame=single]
$ mvn package
			\end{Verbatim}
			Les dépendances suivantes seront téléchargées:
			\begin{itemize}
				\item jsr305 $3.0.1$ (com.google.code.findbugs)
				\item junit $4.4$ (junit)
				\item math $13.0$ (org.arakhne.afc.core)
				\item mathfx $13.0$ (org.arakhne.afc.advanced)
			\end{itemize}
			Et les \textit{jar} suivant seront créé et placé dans \textit{./target}:
			\begin{itemize}
				\item MagicMetro.jar
				\item MagicMetro-jar-with-dependencies.jar
			\end{itemize}
	\section{Fonctionnalités prêtes}
		\p{}
		Les fonctionnalités que nous avons réussi à implémenter dans les temps sont les suivantes:
		\begin{itemize}
			\item Menus avec un système du type pushdown automaton pour revenir en arrière (touche échap et bouton Exit)
			\item Menu d'option avec choix du full screen
			\item Menu de choix des maps dynamique en fonction des MapScript présents
			\item Menu pour quitter une partie en cour (touche échap)
			\item Choix entre plusieurs Maps
			\item Système de scripts de description de maps
				\begin{itemize}
					\item Apparition des stations (moment et position)
					\item Choix des bonus (moment, type et nombre)
				\end{itemize}
			\item Gestion du temps (play, pause, vitesse)
			\item Inventaire
			\item Ajout de trains aux lignes depuis l'inventaire
			\item Amélioration des station avec les amélioration présentes dans l'inventaire
			\item Path-Finding pour les passagers
			\item Gestion des lignes
				\begin{itemize}
					\item Création de nouvelle ligne
					\item Extension de ligne par l'extrémité
					\item Extension de ligne par une section
					\item Régression de ligne par l'extrémité
				\end{itemize}
			\item Gestion de plusieurs skin graphique
		\end{itemize}
	\section{Fonctionnalités à implémenter}
		\p{}
		Faute de temps, nous n'avons pas réussi à implémenter ces fonctionnalités:
		\begin{itemize}
			\item Gestion des Wagons (Vue et modèle)
			\item Retrait de Train d'une ligne
			\item Possibilité de choisir dans quel sens partira le train lorsqu'on le pose depuis l'inventaire
			\item Mouvement circulaire des trains quand la ligne est circulaire.
			\item Skins secondaires (mode nuit, mode daltonien\ldots)
		\end{itemize}
\end{document}
