\documentclass[report, backcover, french, nodocumentinfo]{upmethodology-document}
\include{settings_final}
% For more information about UPmethodology: https://www.ctan.org/pkg/upmethodology

\begin{document}

	\upmdocumentsummary{}
	\upmdocumentauthors{}
	\upmdocumentinformedpeople{}
	\upmpublicationpage{}

	\newpage{}
	\section{Etat général}
		\p{}
			Le projet utilise \textit{Apache Maven} pour compiler, un \textit{jar} classique et un contenant les dépendances peuvent être créés avec la commande:
			\begin{Verbatim}[frame=single]
$ mvn package
			\end{Verbatim}
			Les dépendances suivantes seront téléchargées:
			\begin{itemize}
				\item jsr305 $3.0.1$ (com.google.code.findbugs)
				\item junit $4.4$ (junit)
				\item math $13.0$ (org.arakhne.afc.core)
				\item mathfx $13.0$ (org.arakhne.afc.advanced)
			\end{itemize}
			Et les \textit{jar} suivant seront créé et placé dans \textit{./target}:
			\begin{itemize}
				\item MagicMetro.jar
				\item MagicMetro-jar-with-dependencies.jar
			\end{itemize}
	\section{Fonctionnalité présents dans le projet}
		\p{}
		Les fonctionnalités que nous avons réussi à implémenter dans les temps sont les suivantes : 
		\begin{itemize}
			\item Gestion des Lignes et présences de plusieurs ligne.
			\item Choix entre plusieurs Maps
			\item Présence d'un menu option pour mettre en pleine écran
			\item Ajout de Train
			\item Apparition de station en fonction du temps
			\item Path-Finding pour les passengers
			\item Upgrade des stations (Bonus)
			\item Choix de Bonus tous les X temps.
		\end{itemize}
	\section{Fonctionnalité à Implémenter}
		\p{}
		Faute de temps, nous n'avons pas réussi à implémenter ces fonctionnalités : 
		\begin{itemize}
			\item Gestion des Wagons (Vue et modèle)
			\item Retrait de Train d'une ligne
			\item Loopage des trains quand la ligne est circulaire.
		\end{itemize}
\end{document}
